\section{Business Requirements}
\subsection{Background}
% What is the business context for the project?
Teachers and professors spend a large amount of time grading tests and other documents. The product that we will deliver hopes to hasten the throughput rate for this process.
% What does the reader need to know about the customer's industry?

\subsection{Business Opportunity}
% What is the customer's need?
Our customers want to have a grading tool that is more efficient than doing it by hand. While such tools already exist there are some disadvantages to them, such as having to upload files up to their systems and losing the rights to the documents that you upload.
% How does this need fit in the industrial context?
not yet certain.
\subsection{Business Objectives and Success Criteria}
% Why does the customer need this product?
The customer needs our product because current methods of grading consist of either manually grading tests or using an equivalent on-line tool with possessive licensing agreements.
% How will the customer know that the product is a success?
Our customer will know that the product was a success when they spend less time overall grading tests. Our product intends to be a more efficient means of assessing documents then just by hand and so if it even scrapes off an hour of grading time then the product would be considered successful.
%   You must be very specific here.  What experiment can we do to verify success?
In less ambiguous terms, while using our product it should take under four hours to finish grading a final assessment and should be at least an hour quicker then doing it by hand.
\subsection{Customer and Market Needs}
% Connect the objectives and success criteria back to the customer's business.
% That is, why does the industry require the customer to have this solution? 
Our customers could have dozens of students under their tutelage which leads to having reams of pages of tests to grade. Our product hopes to decrease the amount of time spent going through all of those pages, thus allowing our customer to return feedback to their students at a quicker rate by making grading more efficient so that grades can be handed back sooner. This should help diminish the common complaint in academic settings that the time frame between handing in an exam and receiving the results is much too long.
\subsection{Business Risks}
% This is a hazard assessment.  What could go wrong?  How bad would it be if it
% did?  How likely is it?  What steps can we take to protect against the hazard? 
Our product is going to be written to work in the Linux environment and so should our clients suddenly shift over to a different OS it would be disastrous for our product. This risk however, is not very likely, our current client is almost certain to continue using Linux well beyond the completion of the product. Future clients may want our product to be available on different platforms and so in the future we may port our program to those platforms.
