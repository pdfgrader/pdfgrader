
\section{Business Context}

\subsection{Stakeholder Profiles}
% This section describes who will benefit from the product both directly and
% indirectly.  Identify stake holders by their roll and describe their needs and
% interactions for that role.

The stakeholders for this product include everyone involved in the grading of work. This includes Professors, Teachers, TAs, and Students. Anyone grading work will be directly interacting with the product and its features. They need an efficient way to get grading done, and this product will attempt to do that. Stakeholders also include the university and department depending on interest.
\newline



\begin{tabular}{|l|l|l|}
\hline
\textbf{Stakeholder} & \textbf{Major Value}                                                    & \textbf{Major Interests}                                                                             \\ \hline
Graders              & Efficiency                                                              & \begin{tabular}[c]{@{}l@{}}Less legal fuss with\\ existing solutions and\\ time saving\end{tabular}  \\ \hline
Students             & \begin{tabular}[c]{@{}l@{}}Speed of \\ feedback\end{tabular}            & \begin{tabular}[c]{@{}l@{}}speed of grading process and\\ custom feedback\end{tabular}              \\ \hline
Institution          & \begin{tabular}[c]{@{}l@{}}Benefits Students\\ and Faculty\end{tabular} & \begin{tabular}[c]{@{}l@{}}Streamlines grading\\ and improves productivity\end{tabular}              \\ \hline
\end{tabular}



\subsection{Project Priorities}
% Prioritize the major features identified above based on the relative
% return-on-investment.  

The major features that have the highest priority are page ordering, view, ability to give comments, ability to deduct points, and ability to grade in parallel with multiple graders. These features will be prioritized in development to ensure they are present in the first release.
\newline

\begin{tabular}{|l|l|}
\hline
\textbf{Priority} & \textbf{Feature}                                                                                                                                                                                              \\ \hline
High              & \begin{tabular}[c]{@{}l@{}}Ability to view PDF documents question by question \\ Ability to grade and comment questions in a desired output format\\ Linux Compatibility\end{tabular} \\ \hline
Medium Priority   & \begin{tabular}[c]{@{}l@{}}Ability to grade in parallel while minimizing conflicts\\ Ability to set and use hotkeys while grading\end{tabular}                                                                \\ \hline
Low Priority      & Name or question detection/vision                                                                                                                                                                             \\ \hline
\end{tabular}

\pagebreak
\subsection{Operating Environment}
% Describe how and where the product will be used.  

The operating environment of out product will be mainly aimed towards Teachers and Professors in their preferred grading environment. As long as the user has access to the PDF documents to be graded, the environment should be one that the grader prefers to do their grading in. The product should increase their efficiency in grading giving them a better experience overall.