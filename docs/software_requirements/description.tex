\section{Overall Description}
% Describe the product's context in the larger business or industry setting.  Do
% not include specific features.  Give the reader an understand of how those
% features fit into the larger setting. 
%
PDF Grader is intended to be a program that assists grading large sets of tests.
It will allow the grader to grade tests faster and more consistently than if they were to grade them manually by hand.
The graders will also be able to provide students with feedback to their individual test.

\subsection{Product Perspective}
% How does the product fit in the business larger processes.  How is the user
% intended to fit the software into their business activities?   Consider
% including a figure that illustrates these relationships. 
Our product is a replacement for manual grading that fits inbetween the current workflow of grading each test manually.

\subsection{Product Features}
% This is a list of high-level description of the functional behavior of the
% product.  This should give the reader a better understanding of how the formal
% requirements fit together.
\begin{itemize}
    \item Allows for fast grading and concurrent grading on the same paper to speed up grading.
    \item Allows the user to upload a pdf file to the program, which will then ask for some information about the test format.
    \item Scroll to the same question for each test so that a single question can be graded for the entire set quickly.
    \item Asks if there is a rubric that could be "hotkeyed" for speeding up the grading rate per question.
    \item Generates a set of final results that can be given out to students so that they can see the errors and feedback at the same time.
\end{itemize}
%PDF Grader allows for fast grading and concurrent grading on the same paper to speed up grading.
%It will do so by allowing the user to upload a pdf file to the program, which will then ask for some information about the test format.
%Using the format information it will be able to scroll to the same question for each test so that a single question can be graded for the entire set quickly.
%The software will also ask if there is a rubric that could be hotkeyed for speeding up the grading rate per question.
%After grading has been completed the program will also generate a set of final results that can be given out to students so that they can see the errors and feedback at the same time.

\subsection{User Classes and Characteristics}
% Describe the different rolls or classes of users.  For each user class,
% describe the user class's principal characteristics.  For example, unix
% systems have at least two classes of users: system administrators and
% operators.
There are two roles: Main grader and TA. \par 
The main grader role is the role that will have access to all of the features. \par
The TA role will have access to all features except being able to generate the results pages and exporting the final graded tests.
%The main role will have access to all of the features. Another role will have certain features restricted, intended for use by TA's.

\subsection{Operating Environment}
% What is the expected environment?  For example, the product could be a desktop
% application with users who work in a formal office environment.  Contrast this
% with a mobile application for mountain biking that keeps track of GPS locations.
The product is intended to be used wherever the user has a desktop computer.

\subsection{Design and Implementation Constraints}
% List an constraints that are part of the project.  For example, health
% services applications must implement HIPAA regulations.  
PDF Grader must be runnable in the Linux environment.

\subsection{Assumptions and Dependencies}
% List assumptions and dependencies that are not formal constraints.  Items in
% this list will, if changed, will cause a change in the formal requirements in
% the next section.
%
We assume that our client will continue using the Linux environment, and, for the concurrency aspect to function, using a file sharing system.
