\section{Other Nonfunctional Requirements}
% Nonfunctional requirements are shall-statements about how the software
% performs or is written.  It is not a statement about what the software does.
% For example, the function requirements for a Fibonacci function is this: the
% function shall return the nth Fibonacci number when provided n as input.  A
% nonfunctional requirement would be these: the function shall take time O(n)
% and all lines of the software shall be reachable by some test-case.
%

\subsection{Performance Requirements}
% Your project will have performance requirements.  How long is the user willing
% to wait for the different features to execute?   For example, the software
% shall authenticate authorized users withing 250ms.  

The software shall be able to load a 100 page pdf document in 10 seconds or less.
The software shall allow the grader to grade a set of finals tests in one hour less than when not using the software.

\subsection{Safety Requirements}
% It is unlikely that your project will have any safety requirements.  This
% section would be used for software that controls a physical device that could
% potentially cause harm.  
None at this time.

\subsection{Security Requirements}
% Do not overlook this section.  Consider the three primary topics of computer
% security: Confidentiality, Integrity, and Availability.   Your project will
% have security requirements. 
%

The software shall store all data files accosiated with grading information in encrypted format. 
The software shall require that the user use a password to open saved files. 
The software shall hide the names on the documents from TAs. 
The software shall only communicate with the file share that the document is located.

\subsection{Software Quality Attributes}
% Quality requirements can be the most difficult to write.  The desired quality
% must be written in a meaningful and testable way.  For example, All functions
% within the software shall have a cyclomatic number less than 10.  

The software shall be efficient enough that a 45\% difference is made between traditional grading and software assisted grading. The software interface shall be intuitive enough that given a 30 minute instruction a professor or perspective grader should be able to use the software proficiently. 

